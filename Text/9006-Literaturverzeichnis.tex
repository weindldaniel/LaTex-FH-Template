\chapter{Literaturverzeichnis}
\label{sec: Bibliography}
Nachstehend finden Sie Muster-Literaturverzeichnisse, die die wichtigsten Arten von Veröffentlichungen abdecken (reine Internet-Quelle, Norm, Monografie, Aufsatz in einem Sammelband, technische Regel, Zeitschriftenaufsatz, KI-Tool). Informationen zu weiteren Publikationstypen finden Sie für das Deutsche in der unten genannten ONR~12658, für das Englische in der unten genannten ISO~690. Diese Regelwerke sind in der Bibliothek des FH-OÖ-Campus Wels bzw. online einsehbar.

\setlength{\parindent}{0mm}
\section{Literaturverzeichnis bei Nutzung von Quellenangaben im Text (AGR, AT, MB, PDK) oder in Fußnote (AMM, MEWI)}

\printbibliography[heading=none,env=bibliographyAlpha]
\nocite{*}

\vspace{1mm}
\fbox{\parbox{\textwidth}{Vordefinierte Zitierstile, die dieser Darstellung nahekommen, sind in Literaturverwaltungsprogrammen mit Stichwörtern wie „ISO 690“ zu finden. Die Verwendung solcher Zitierstile erfordert unter Umständen manuelle Anpassungen vor Abgabe Ihrer wissenschaftlichen Arbeit.}}

\section{Literaturverzeichnis bei Nutzung von Quellenangaben im Text (BI, BUT, IPM, LCW, LTE)}

Austrian Standards Institute (2013), \textit{ONR 12658:2013 Empfehlungen zum Zitieren von Informationsquellen und Anleitungen zur Gestaltung von Literatur- und anderen Quellennachweisen in wissenschaftlichen Arbeiten.}

\vspace{1mm}
Austrian Standards International (2023) \textit{"ONORM A 2662:2023 Wissenschaftliche Abschlussarbeiten – Angaben für den bibliographischen Nachweis.}

\vspace{1mm}
Beuermann, C. (2013) \textit{Die Entdeckung des menschlichen Einflusses auf das Klima.} Verfügbar von: 
\texttt{https://www.bpb.de/gesellschaft/umwelt/klimawandel/38444/}
\texttt{entdeckung-des-menschlichen-einflusses} (abgerufen am: 14.07.2021).

\vspace{1mm}
International Organization for Standardization (2021) \textit{ISO 690:2021 Information and documentation – Guidelines for bibliographic references and citations to information resources}. 

\vspace{1mm}
Karmasin, M. und Ribing, R. (2019) \textit{Die Gestaltung wissenschaftlicher Arbeiten: Ein Leitfaden für Facharbeit/VWA, Seminararbeiten, Bachelor-, Master-, Magister- und Diplomarbeiten sowie Dissertationen.} 10. Aufl. Wien: facultas.

\vspace{1mm}
Lehrndorfer, A. und Reuther, U. (2008) ‚Kontrollierte Sprache – standardisierte Sprache?{‘}, in Muthig, J. (Hrsg.) \textit{ Standardisierungsmethoden für die Technische Dokumentation.} Lübeck: Schmidt-Römhild, S. 97 121.

\vspace{1mm}
Meier, J. (2011) \textit{Globalisierung}. Wiesbaden: Pons.

\vspace{1mm}
Müller, E. und Meier, J. (2019) \textit{Globalisierung neu gedacht.} Wiesbaden: Pons.

\vspace{1mm}
Müller, E., Meier, J. und Huber, A. (2021) \textit{Globalisierung: Rahmenbedingungen, Prozesse, Institutionen.} Wiesbaden: Pons.

\vspace{1mm}
Müller, E., Meier, J., Huber, A. und Tausch, G. (2016) \textit{Globalisierung: Gegenwart und Zukunft.} Wiesbaden: Pons.

\vspace{1mm}
OpenAI (Hrsg.) (2023) \textit{ChatGPT, Version xy} [Sprachmodell]. Antwort auf die Prompt-Eingabe der Autorin/des Autors „xyz“, erzeugt am 01.12.2023. Verfügbar von:
\newline \texttt{https://chat.openai.com/} (abgerufen am: 01.12.2023).

\vspace{1mm}
Schulz, M. (2014) ‚Doku-Norm in der Praxis‘, \textit{technische kommunikation}, 36(6), \linebreak S. 46-49.

\vspace{1mm}
\fbox{\parbox{\textwidth}{Vordefinierte Zitierstile, die dieser Darstellung nahekommen, sind in Literaturverwaltungsprogrammen mit Stichwörtern wie „cite them right“ zu finden. Die Verwendung solcher Zitierstile erfordert unter Umständen manuelle Anpassungen vor Abgabe Ihrer wissenschaftlichen Arbeit.}}

\section{Literaturverzeichnis bei Nutzung von Quellenangaben in Form von Zahlen in eckigen Klammern (AB, AET, AGR, AMM, AT, BUT, EE, LTE, MB, MEWI, RSE, SES, VTP, WFT)}

\printbibliography[heading=none]

\vspace{1mm}
\fbox{\parbox{\textwidth}{Vordefinierte Zitierstile, die dieser Darstellung nahekommen, sind in Literaturverwaltungsprogrammen mit Stichwörtern wie „ISO 690“ zu finden. Die Verwendung solcher Zitierstile erfordert unter Umständen manuelle Anpassungen vor Abgabe Ihrer wissenschaftlichen Arbeit.}}


\subsection*{Erscheinungsbild}
Das Aussehen des Literaturverzeichnisses hängt von der Zitierweise ab, die Sie in Ihrer wissenschaftlichen Arbeit anwenden müssen [siehe Abschnitt \ref{sec: ZitierStil} auf S. \pageref{sec: ZitierStil}].

\subsection*{Gliederung}
Das Literaturverzeichnis kann bei Bedarf in folgende Unterkapitel untergliedert werden:
\begin{itemize}
	\item	Primärliteratur 
	\item	Sekundärliteratur 
	\item	Tertiärliteratur
\end{itemize}

\subsection*{Literaturrecherche}
Auf den Internetseiten der Bibliothek des FH-OÖ-Campus Wels finden Sie eine sehr übersichtliche Liste mit zahlreichen Links zu:
\begin{itemize}
	\item	elektronischen Zeitschriften
	\item	Datenbanken
	\item	zahlreichen Bibliotheken
	\item	Katalogen des Buchhandels
	\item	Patentgesellschaften
\end{itemize}
Die Mitarbeiter*innen der Bibliothek unterstützen Sie gerne bei der Literaturrecherche.

