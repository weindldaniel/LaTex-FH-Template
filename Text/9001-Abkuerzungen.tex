\chapter{Abkürzungsverzeichnis [optional]}
\label{sec: Abkürzung}

\begin{table}[h]
	\centering
	\begin{tabular}{|p{3cm}|p{11.6cm}|}
		\hline
		\rowcolor{lightgray}\textbf{Abkürzung} & \textbf{Erklärung} \\
		\hline
		 & \\ 
		\hline
		 & \\ 
		\hline
		 & \\ 
		\hline
	\end{tabular}
\end{table}

\noindent Das Abkürzungsverzeichnis kann bei einer Vielzahl von Abkürzungen verwendet werden. Sortieren Sie das Abkürzungsverzeichnis nach Fertigstellung in alphabetischer Reihenfolge.

%%%+++++++++++++++++++++++++++++++++++++++++++++++++++++++++++++++++++++++++++
%% Latex bietet auch die Möglichkeit eines automatischen Abkürzungsverzeichnisses. Dieses entspricht aber nicht der Vorgabe und ist somit mit dem Betreuer abzusprechen.
%% In den folgenden Zeilen sind die nötigen Einstellungen angegeben und der Einfügebefehl angeführt.
%%%---------------------------------------------------------------------------
%% automatisches Abkürzungsverzeichnis
%%%---------------------------------------------------------------------------
%% In den TeXstudio-Einstellungen: 
%% ("Zeige erweiterte Einstellungen" muss aktiviert sein (In Einstellungsfenster: unten links))
 
%% 1. Schritt: Makeindex konfigurieren 
%%	- Reiter Befehle: 	Makeindex: makeindex.exe %.nlo -s nomencl.ist -o %.nls
 
%% 2. Schritt: Makeindex beim kompilieren ausführen 
%%	- Reiter Erzeugen:	Standardkompiler (rechts auf Schraubenschlüssel):
%%						makeindex zufügen 
%%%---------------------------------------------------------------------------

%% Abkürzungen hier einfügen:
%\abbrev{FH}{Fachhochschule}
%\abbrev{AT}{Automatisierungstechnik}

%\vspace{-41.5mm}
%\begin{minipage}{\textwidth}
%	\renewcommand{\nomname}{}
%	\printnomenclature
%\end{minipage}
%%%+++++++++++++++++++++++++++++++++++++++++++++++++++++++++++++++++++++++++++